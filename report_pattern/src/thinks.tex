\section{Выводы}
В ходе выполнения лабораторной работы была успешно реализована программа для
поиска всех вхождений образца в текст с использованием статистики совпадений.
\begin{itemize}
    \item Освоена структура суффиксного дерева и принципы его построения с помощью онлайн-алгоритма Укконена, включая ключевые концепции, такие как
суффиксные ссылки и "активная точка".
    \item Реализован метод решения задачи на основе обобщенного суффиксного дерева.d
    \item Реализован поиск подстроки в строке с использованием статистики совпадений, что позволило выполнить задачу за линейное время.
    \item Проведено сравнительное тестирование производительности, которое на практике подтвердило асимптотическое преимущество линейного алгоритма на суффиксных деревьях (O(N)) над квадратичным решением на основе динамического программирования (O($N^2$
))
    \item Получен практический опыт работы со сложными динамическими структурами данных на C++, включая ручное управление памятью и указателями, что
является важным для понимания низкоуровневых аспектов программирования.
Разработанное решение является эффективным и масштабируемым, а полученные
знания о суффиксных структурах данных могут быть применены для решения широкого круга других задач в области биоинформатики, обработки текстов и анализа
данных.
\end{itemize}.
\pagebreak
